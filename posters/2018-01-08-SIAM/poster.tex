\documentclass[usenames,dvipsnames,t]{beamer}

\usepackage[english]{babel}
\usepackage[utf8]{inputenc}
\usepackage{amsmath,amsthm, amssymb, latexsym}
\usepackage{color}
\usepackage{tikz}
\usepackage{standalone}
\usepackage{minted}
\usepackage{lmodern}

\definecolor{DarkGray}{RGB}{5, 66, 81}
\definecolor{DarkerGray}{RGB}{3, 22, 27}

\usemintedstyle{native}

\usetikzlibrary{decorations.pathmorphing}
\usetikzlibrary{decorations.pathreplacing,angles,quotes}

\usecolortheme[dark,accent=cyan]{solarized}
\beamertemplatenavigationsymbolsempty
\setbeamerfont{block title}{size=\Large}
\usepackage[orientation=portrait,size=a0,scale=1.4]{beamerposter}


%%%%%%%%%%%%%%%%%%%%%%%%%%%%%%%%%%%%%%%%%%%%%%%%%%%%%%%%%%%%%%%%%%%%%%%%%%%%%%%
\begin{document}

%%%%%%%%%%%%%%%%%%%%%%%%%%%%%%%%FIRST ROW%%%%%%%%%%%%%%%%%%%%%%%%%%%%%%%%%%%%%%%
\begin{columns}
    \begin{column}{.025\linewidth}
    \end{column}
    \begin{column}{.5\linewidth}
        \vspace{0.7cm}

        \centering
        \textcolor{orange}{\fontsize{180}{400} \selectfont THE POWER}
        \vspace{0.6cm}

        \textcolor{orange}{\fontsize{180}{400} \selectfont OF MEMORY}
        \vspace{0.6cm}

        \Large\textcolor{orange}{In interactions both social and biological can memory
        be advantageous?}
    \end{column}
    \begin{column}{.025\linewidth}
    \end{column}
    \vrule width 10pt

    \begin{column}{.025\linewidth}
    \end{column}

    \begin{column}{.4\linewidth}
        \vspace{2cm}

    \begin{itemize}
        \item Both players are better of choosing Cooperation (3)
        \item there is always a temptation for a player to Defect (5).
    \end{itemize}

        \large{
        \begin{equation}\label{eq:payoff_matrix_symb}
             \bordermatrix{~ & C & D \cr
                              C & (3, 3) & (5, 0) \cr
                              D & (0, 5) & (1, 1) \cr}
            \end{equation}
        }
    \end{column}

    \begin{column}{.025\linewidth}
    \end{column}
\end{columns}

%%%%%%%%%%%%%%%%%%%%%%%%%%%%%%%SECOND ROW%%%%%%%%%%%%%%%%%%%%%%%%%%%%%%%%%%%%%%%
\begin{columns}
    \begin{column}{.025\linewidth}
    \end{column}
    \begin{column}{.5\linewidth}
        \vspace{0.9cm}
    
   \small{In the 1980's, Robert Axelrod carried out a computer tournament of the
          iterated prisoner's dilemma. In the iterated version of the game the players
          interact for an infinite number of time and they have access to the full history
          of the matches. Axelrod's results argued for the first time how
          cooperation can be evolutionarily advantageous.}
    
    \begin{center}
        \includestandalone[width=0.6\textwidth]{static/memory_one}
    \end{center}

    \small{In 2012 Press and Dyson studied the iterated prisoner's dilemma in a
           similar manner. They stated that in a two players interaction, a player
           with the shortest memory in effect sets the rules of the game. A player
           with a good memory-one strategy can force the game to be played,
           effectively, as memory-one. Thus, in the iterated prisoner's dilemma,
           memory is not advantageous.}
    \vspace{0.5cm}
    
    \small{The purpose of this work is to consider a given memory one strategy 
    \(q=(q_1, q_2, q_3, q_4)\). However previous work found a way for
    the opponent of \(q\) to manipulate \(q\), this work will consider an optimisation
    approach to identify the best response \(p^*=(p_1, p_2, p_3, p_4)\) to a strategy
    \(q\). In essence answering the question: what is the best memory one strategy
    against a given other memory one strategy.
    }
    \end{column}

    \begin{column}{.5\linewidth}
    \end{column}
\end{columns}

\end{document}

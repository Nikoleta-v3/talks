\documentclass{beamer}

%%%%%%%%%%%%%Solarized Theme%%%%%%%%%%%%%%%
\usecolortheme[dark,accent=cyan]{solarized}
\beamertemplatenavigationsymbolsempty

%%%%%Packages%%%%%
\usepackage{graphicx}
\graphicspath{ {static/} }

\usepackage{hyperref}
\usepackage{colortbl, xcolor}
\usepackage{booktabs}

\usepackage{tikz}
\usepackage{standalone}
\usetikzlibrary{decorations.pathmorphing, decorations.markings, arrows}
\usetikzlibrary{decorations.pathreplacing,angles,quotes, calc}

\usepackage{amsmath}
\usepackage{amsthm}
\usepackage{amssymb}

\setbeamertemplate{blocks}[default]

%%%%%%Title%%%%%%%%
\title{Memory size in the Prisoner's Dilemma}
\author{Nikoleta E. Glynatsi}
\date{\tiny{Supervised by:} \\ \small{Dr. Vincent \textsc{Knight} \hspace{1cm} 
Dr. Jonathan \textsc{Gillard} }}
\institute[]
{
\begin{center}
    \includegraphics[width=.20\textwidth]{cardiff_uni_logo.jpg}
\end{center}
}

\begin{document}

\maketitle

\begin{frame}
    \centering
    \includestandalone[width=.8\textwidth]{matrix}
\end{frame}

\begin{frame}
    \centering
    \includestandalone[width=.7\textwidth]{memory_one}
\end{frame}

\begin{frame}
    \begin{center}
    \textbf{\textcolor{orange}{
    \textit{William H. Press and Freeman J. Dyson.}
    Iterated Prisoner's Dilemma contains strategies that dominate any evolutionary
    opponent. 2012.}}
    \end{center}
\end{frame}

\begin{frame}
    \centering
    \includestandalone[width=0.5\textwidth]{memory_one_chain}
  
    \vfill
    \textcolor{orange}{
    \large
    \boldmath\( p = (p_1, p_2, p_3, p_4) \in\mathbb{R}_{[0,1]}^{4} \)}
\end{frame}

\begin{frame}
    \centering
    \includestandalone[width=.6\textwidth]{states} \vspace{1cm}

    \includestandalone[width=.8\textwidth]{m_matrix}
\end{frame}

\begin{frame}
    \begin{center}
    \large{
    \textbf{\textcolor{orange}{
    WHICH IS THE BEST MEMORY ONE STRATEGY? \\ \vspace{.7cm}
    ARE THEY LIMITATIONS TO MEMORY ONE STRATEGIES?}}}
    \end{center}
\end{frame}

\begin{frame}
    \centering
    \Large\textcolor{orange}{
    \boldmath \( \max_p u_q(p)\text{ such that }p\in\mathbb{R}_{[0,1]}^{4}\)}
\end{frame}

\begin{frame}
    \begin{center}
    \begin{lemma}
     \boldmath\[ u_q(p) = \frac{\frac{1}{2} p Q  p^T + c^T  p + a} 
    {\frac{1}{2} p \bar{Q} p^T + \bar{c}^T p + \bar{a}}\]

    \begin{itemize}
      \item \boldmath\(Q, \bar{Q} \in\mathbb{R}^{4 \times 4}\)
      \item \boldmath\(c, \bar{c}\in\mathbb{R}^{4 \times 1}\) 
      \item \boldmath\(a, \bar{a}\in\mathbb{R}\)  
   \end{itemize}
    \end{lemma}
    \end{center}
\end{frame}

\begin{frame}
    \begin{center}
    \includestandalone[width=.4\textwidth]{cube}
    \end{center}
\end{frame}

\begin{frame}
    \begin{center}
    \includestandalone[width=.4\textwidth]{cube_two}
    \end{center}
\end{frame}

\begin{frame}
    \begin{center}
    \includestandalone[width=.4\textwidth]{cube_three}
    \end{center}
\end{frame}

\begin{frame}
    \begin{center}
    \Large{
    \textbf{\textcolor{orange}{PURELY RANDOM}} \vspace{1cm}

    \textcolor{orange}{\boldmath\( p = (p, p, p, p)\)}}
    \end{center}
\end{frame}

\begin{frame}
    \begin{center}
    \Large{
    \textbf{\textcolor{orange}{REACTIVE}} \vspace{1cm}

    \textcolor{orange}{\boldmath\( p = (p_1, p_2, p_1, p_2)\)}}
    \end{center}
\end{frame}

\begin{frame}
    \begin{center}
    \Large{
    \textbf{\textcolor{orange}{DIFFERENTIAL EVOLUTION}}}
    \end{center}
\end{frame}

\begin{frame}
    \begin{center}
    \includestandalone[width=\textwidth]{limitations}
    \end{center}
\end{frame}

\begin{frame}
\centering
    \large \textbf{\textcolor{orange}{{@NikoletaGlyn}}}\\
    \large \textbf{\textcolor{orange}{{https://github.com/Nikoleta-v3}}}\\
\end{frame}

\end{document}
